\begin{acknowledgements}
Θα ήθελα να ευχαριστήσω τον καθηγητή  κ. Νικόλαο Κανταρτζή για την επίβλεψη αυτής της διπλωματικής εργασίας, τη στήριξή και βοήθειά του καθώς και τις υποδομές που επέτρεψαν την ολοκλήρωση της διπλωματικής στο ΑΠΘ.

Επίσης, ευχαριστώ ιδιαίτερα τον \en{Dr. Adam Jeff}, επιβλέποντα καθηγητή μου στο \en{CERN}, για την καθοδήγησή του και την εξαιρετική συνεργασία που είχαμε τους 12 μήνες που εργαζόμουν στην Ελβετία. 

Τέλος θα ήθελα να ευχαριστήσω τους γονείς μου για την καθοδήγηση, την ηθική και υλική συμπαράσταση που μου προσέφεραν όλα αυτά τα χρόνια.
\end{acknowledgements}

\begin{abstract}
Η περίληψη θα συμπληρωθεί αργότερα. 

   \begin{keywords}
   %3 λέξεις

   \end{keywords}
   
\end{abstract}



\begin{abstracteng}
\tl{The Compact LInear Collider (CLIC) will use a novel acceleration scheme in which energy extracted from a very intense beam of relatively low-energy electrons (the Drive Beam) is used to accelerate a lower intensity Main Beam to very high energy. 
The high intensity of the Drive Beam, with pulses of more than $10^{15}$ electrons, poses a challenge for conventional profile measurements such as wire scanners.
Thus, new non-invasive profile measurements are being investigated.}

\tl{One candidate is the Electron Beam Scanner. A probe beam of low-energy electrons crosses the accelerator beam perpendicularly. 
The probe beam is deflected by the space-charge fields of the accelerator beam. 
By scanning the probe beam and measuring its deflection with respect to its initial position, the transverse profile of the accelerator beam can be reconstructed.}

\tl{Analytical expressions for the deflection exist in the case of long bunches, where the charge distribution can be considered constant during the measurement. 
In this thesis we consider the performance of an electron beam scanner in an accelerator where the bunch length is much smaller than the probe-beam scanning time. 
In particular, the case in which the bunch length is shorter than the time taken for a particle of the probe beam to cross the main beam is difficult to model analytically. We have developed a simulation framework allowing this situation to be modelled.}

   \begin{keywordseng}
    \tl{Fill in }
    %TODO fill in keywords
   \end{keywordseng}

\end{abstracteng}
