\begin{abstract}
Ο \en{Compact LInear Collider (CLIC)} θα χρησιμοποιεί μια καινοτόμα μέθοδο επιτάχυνσης, στην οποία ενέργεια που εξάγεται από μια υψηλής έντασης δέσμη ηλεκτρονίων σχετικά χαμηλής ενέργειας (τη Δέση Οδηγό), χρησιμοποιείται για την επιτάχυνση μιας χαμηλότερης σε ένταση Κύριας Δέσμης σε πολύ υψηλή ενέργεια.
Η υψηλή ένταση της Δέσμης Οδηγού, με παλμούς που περιέχουν πάνω από $10^{15}$ ηλεκτρόνια, αποτελεί τροχοπέδη στη χρήση συμβατικών μεθόδων μέτρησης προφίλ, όπως τους \en{wire scanners}.
Έτσι, εξετάζονται νέες μη επεμβατικές μέθοδοι μέτρησης προφίλ.

Μια υποψήφια μέθοδος είναι ο Σαρωτής Δεσμών Ηλεκτρονίων (\en{Electron Beam Scanner}). 
Μια δέσμη ανίχνευσης ηλεκτρονίων σε χαμηλή ενέργεια διασχίζει κάθετα την δέσμη του επιταχυντή.
Η δέσμη αυτή αποκλίνει από την πορεία της λόγω των πεδίων της δέσμης του επιταχυντή.
Σαρώνοντας τη δέσμη ανίχνευσης και μετρώντας την απόκλιση της σε σχέση με την αρχική της θέση, το εγκάρσιο προφίλ της δέσμης του επιταχυντή μπορεί να ανακατασκευαστεί.

Αναλυτικές εκφράσεις για την απόκλιση υπάρχουν για την περίπτωση δεσμών μεγάλου μήκους, όπου η κατανομή φορτίου μπορεί να θεωρηθεί σταθερή κατά τη μέτρηση.
Στην παρούσα διπλωματική εργασία εξετάζουμε την επίδοση του σαρωτή δεσμών ηλεκτρονίων σε έναν επιταχυντή όπου το μήκος της δέσμης είναι πολύ μικρότερο από τη διάρκεια σάρωσης της δέσμης ανίχνευσης.
Συγκεκριμένα, η περίπτωση όπου το μήκος της δέσμης είναι μικρότερο από το χρόνο που απαιτεί ένα σωματίδιο της δέσμης ανίχνευσης για να διασχίσει την κύρια δέσμη είναι δύσκολο να μοντελοποιηθεί αναλυτικά.
Αναπτύξαμε ένα περιβάλλον προσομοίωσης που επιτρέπει τη μοντελοποίηση αυτής της περίπτωσης.

   \begin{keywords}
   Σαρωτής Δεσμών Ηλεκτρονίων, προφίλ δέσμης, επιταχυντής \en{CLIC}, μη επεμβατική μέτρηση προφίλ
   \end{keywords}
   
\end{abstract}



\begin{abstracteng}
\tl{The Compact LInear Collider (CLIC) will use a novel acceleration scheme in which energy extracted from a very intense beam of relatively low-energy electrons (the Drive Beam) is used to accelerate a lower intensity Main Beam to very high energy. 
The high intensity of the Drive Beam, with pulses of more than $10^{15}$ electrons, poses a challenge for conventional profile measurements such as wire scanners.
Thus, new non-invasive profile measurements are being investigated.}

\tl{One candidate is the Electron Beam Scanner. A probe beam of low-energy electrons crosses the accelerator beam perpendicularly. 
The probe beam is deflected by the space-charge fields of the accelerator beam. 
By scanning the probe beam and measuring its deflection with respect to its initial position, the transverse profile of the accelerator beam can be reconstructed.}

\tl{Analytical expressions for the deflection exist in the case of long bunches, where the charge distribution can be considered constant during the measurement. 
In this thesis we consider the performance of an electron beam scanner in an accelerator where the bunch length is much smaller than the probe-beam scanning time. 
In particular, the case in which the bunch length is shorter than the time taken for a particle of the probe beam to cross the main beam is difficult to model analytically. 
We have developed a simulation framework allowing this situation to be modelled.}

   \begin{keywordseng}
    \tl{Electron Beam Scanner, beam profile, CLIC accelerator, linear accelerator, non-intercepting profile measurement}
   \end{keywordseng}

\end{abstracteng}
