\chapter{Εισαγωγή}


\epigraph{\textit{\en{``It seems probable to me that God, in the beginning, formed matter in solid, massy, hard, impenetrable, moveable particles\ldots "}}}{\en{---Isaac Newton, \textit\{Opticks (1730)}}}

%Be sure to include a hook at the beginning of the introduction. This is a statement of something sufficiently interesting to motivate your reader to read the rest of the paper, it is an important/interesting scientific problem that your paper either solves or addresses. You should draw the reader in and make them want to read the rest of the paper.




\section{Κίνητρο}
Κίνητρο για τη 

\section{Αντικείμενο της διπλωματικής}
%The next paragraphs in the introduction should cite previous research in this area. It should cite those who had the idea or ideas first, and should also cite those who have done the most recent and relevant work. You should then go on to explain why more work was necessary (your work, of course.) 


\section{Στόχοι της διπλωματικής}
% A statement of the goal of the paper: why the study was undertaken, or why the paper was written. Do not repeat the abstract. 



%%In order to investigate the effect of the short bunches and to optimise the Electron Beam Scanner design, the student should carry out a full analysis, tracking the movement of the probe beams through the electromagnetic fields of the Drive Beam and the surrounding components. The CST simulation suite is the recommended tool for these simulations as it is able to combine electromagnetic and particle-tracking abilities.
%In addition, the student will explore the different options for the probe beam detection and read out, and potentially develop an algorithm for the reconstruction of the Drive Beam profile.
%The student will have the opportunity to contribute to the design of the next generation of particle accelerators. The student will become an expert in the use of the CST simulation suite or equivalent programs. In addition, they will gain an understanding of accelerator beam dynamics and instrumentation design. 




\section{Μεθοδολογία}

CST MATLAB κλπ
σ

\section{Διάρθρωση}
Η παρούσα διπλωματική εργασία είναι οργανωμένη σε πέντε κεφάλαια:

Στο Κεφάλαιο 2 δίνεται το θεωρητικό υπόβαθρο της εργασίας. 
Αυτό περιλαμβάνει μια σύντομη περιγραφή του \en{CERN}, στο οποίο έγινε το μεγαλύτερο κομμάτι της εργασίας, του \en{CLIC}, που αποτελεί τον γραμμικό επιταχυντή που σχεδιάζεται να δημιουργηθεί σε χρονικό ορίζοντα 15 ετών, και παρουσιάζεται και επεξηγείται η λειτουργία του Σαρωτή Δεσμών Ηλεκτρονίων, ή \en{Electron Beam Scanner} όπως θα αναφέρεται.
Σε αυτό το κεφάλαιο αναλύεται και η λειτουργία του \en{Electron Beam Scanner} από μαθηματική σκοπιά, όπως έχει μελετηθεί από την επιστημονική κοινότητα ως τώρα.

Στο Κεφάλαιο 3 παρουσιάζονται οι μέθοδοι προσομοίωσης που χρησιμοποιήθηκαν.
Αρχικά έχουμε μια σύντομη περιγραφή των βασικών εργαλείων που χρησιμοποιήθηκαν, του \en{CST} και του \en{MATLAB}, και στη συνέχεια επεξηγούνται τα στάδια της ανάλυσης που έγινε: ο τρόπος διερεύνησης της επιρροής παραμέτρων του επιταχυντή σε έναν \en{Electron Beam Scanner}, η προσομοίωση της διάταξης εξολοκλήρου στο προσομοιωτικό περιβάλλον του \en{CST} και, τέλος, ο συνδυασμός των δύο εργαλείων, του \en{CST} και του \en{MATLAB}, με σκοπό τη βελτίωση της απόδοσης της προσομοίωσης και της δημιουργίας νέων δυνατοτήτων για προσομοίωση.

Στο Κεφάλαιο 4 παρουσιάζονται και σχολιάζονται τα αποτελέσματα της μελέτης που περιγράφηκε προηγουμένως.

Τέλος, στο Κεφάλαιο 5 δίνονται τα συμπεράσματα, η συνεισφορά αυτής της
διπλωματικής εργασίας, καθώς και προτάσεις για μελλοντική επέκταση της εργασίας.

Στο τέλος της διπλωματικής εργασίας βρίσκονται και 2 παραρτήματα.
Το Παράρτημα Α΄ περιέχει τις μεταφράσεις ξένων όρων που χρησιμοποιήθηκαν στη διπλωματική εργασία, ενώ στο Παράρτημα Β΄ περιγράφεται αναλυτικά το μοντέλο που στήθηκε στο \en{CST}, σε περίπτωση που ο αναγνώστης θελήσει να το αναπαράγει.