\chapter{Εισαγωγή}

\section{Κίνητρο}


\section{Αντικείμενο της διπλωματικής}


\section{Στόχοι της διπλωματικής}
%%In order to investigate the effect of the short bunches and to optimise the Electron Beam Scanner design, the student should carry out a full analysis, tracking the movement of the probe beams through the electromagnetic fields of the Drive Beam and the surrounding components. The CST simulation suite is the recommended tool for these simulations as it is able to combine electromagnetic and particle-tracking abilities.
%In addition, the student will explore the different options for the probe beam detection and read out, and potentially develop an algorithm for the reconstruction of the Drive Beam profile.
%The student will have the opportunity to contribute to the design of the next generation of particle accelerators. The student will become an expert in the use of the CST simulation suite or equivalent programs. In addition, they will gain an understanding of accelerator beam dynamics and instrumentation design. 


\section{Μεθοδολογία}

\section{Διάρθρωση}
Η διπλωματική αυτή είναι οργανωμένη σε πέντε κεφάλαια: Στο Κεφάλαιο 2
δίνεται το θεωρητικό υπόβαθρο των βασικών τεχνολογιών που
σχετίζονται με το αντικείμενο αυτής.
Αρχικά περιγράφονται ..., στη συνέχεια το ... και τέλος ... . 

Στο Κεφάλαιο 3 αρχικά παρουσιάζεται η ανάλυση και η σχεδίαση του συστήματος ... .. 
Τέλος στο Κεφάλαιο 5 δίνονται τα συμπεράσματα, η συνεισφορά αυτής της
διπλωματικής εργασίας, καθώς και μελλοντικές επεκτάσεις.
