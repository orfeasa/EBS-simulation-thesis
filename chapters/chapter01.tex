\chapter{Εισαγωγή}
%Εισαγωγικά, γιατί διάλεξα CST

\begin{enumerate}
\item \en{about CERN}
\item \en{about CLIC}
\end{enumerate}

Το \en{CERN} μπλα μπλα μπλα.

Το \en{CLIC} μπλα μπλα μπλα.

Μέχρι στιγμής οι τρόποι ανίχνευσης μπλα μπλα μπλα.
Μη επεμβατικοί τρόποι πρέπει να πάρουν θέση. Ένας είναι το \en{Electron Beam Scanner}. Παρόλα αυτά μικρή δέσμη στο \en{CLIC}. 

\section{Αντικείμενο της διπλωματικής}
Σκοπός είναι να δούμε αν μπορούμε να χρησιμοποιήσουμε τον \en{Electron Beam Scanner} για να πάρουμε την εικόνα της δέσμη του \en{CLIC}.

\section{Οργάνωση του τόμου}
Η εργασία αυτή είναι οργανωμένη σε πέντε κεφάλαια: Στο Κεφάλαιο 2
δίνεται το θεωρητικό υπόβαθρο των βασικών τεχνολογιών που
σχετίζονται με τη διπλωματική αυτή. Αρχικά περιγράφονται ..., στη συνέχεια το ... και τέλος ... . 
Στο Κεφάλαιο 3 αρχικά παρουσιάζεται ανάλυση και η σχεδίαση του συστήματος ... .. Τέλος στο Κεφάλαιο 5 δίνονται τα συμπεράσματα, η συνεισφορά αυτής της
διπλωματικής εργασίας, καθώς και μελλοντικές επεκτάσεις.
