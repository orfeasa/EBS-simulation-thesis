\chapter{Εισαγωγή}


\epigraph{\textit{\en{``It seems probable to me that God, in the beginning, formed matter in solid, massy, hard, impenetrable, moveable particles\ldots "}}}{\en{---Isaac Newton, \textit{Opticks, 1730}}}

\section{Αντικείμενο της διπλωματικής εργασίας}
Ο επιταχυντής \en{CLIC} αποτελεί τον προτεινόμενο γραμμικό επιταχυντή επόμενης γενιάς.
Η καινοτόμα μέθοδος επιτάχυνσης με δύο δέσμες προϋποθέτει μια Δέσμη Οδηγό πολύ υψηλής έντασης.
Λόγω της υψηλής αυτής έντασης, επεμβατικές συσκευές μέτρησης της χωρικής έντασης (προφίλ) όπως το \en{wire scanner}\footnote{Το \en{wire scanner} είναι μια ηλεκτρο-μηχανική συσκευή η οποία μετρά το προφίλ μιας δέσμης σε έναν επιταχυντή σωματιδίων χρησιμοποιώντας ένα κινούμενο λεπτό σύρμα.
Καθώς το σύρμα περνά μέσα από τη δέσμη, η αλληλεπίδραση δημιουργεί αλληλουχία δευτερευόντων σωματιδίων, τα οποία με τους κατάλληλους αισθητήρες γίνονται αντιληπτά και επιτρέπουν τη ζητούμενη μέτρηση.} θα καταστρέφονταν.
Έτσι, σχεδιάζονται νέες μη επεμβατικές μέθοδοι για να καλύψουν το κενό αυτό.

Μια τέτοια μέθοδος είναι ο Σαρωτής Δεσμών Ηλεκτρονίων, ή εφεξής \en{``Electron Beam Scanner"}, όπου μία ή περισσότερες δέσμες ανίχνευσης στέλνονται κάθετα προς τη Δέσμη Οδηγό.
Αναλύοντας την απόκλιση αυτών των δεσμών σωματιδίων μας δίνεται η δυνατότητα να υπολογίσουμε το προφίλ της Δέσμης Οδηγού.

Τέτοιες συσκευές έχουν χρησιμοποιηθεί στο παρελθόν σε άλλους επιταχυντές που είχαν συνεχόμενη ροή σωματιδίων, ή πολύ μεγάλες δέσμες.
Η Δέσμη Οδηγός του \en{CLIC} θα έχει δέσμες μήκους μόλις 12 \en{picoseconds}\footnote{Το μήκος της δέσμης σε επιταχυντές μπορεί να εκφραστεί και σε μονάδες χρόνου (π.χ.\ \si{\pico \second}), καθώς οι δέσμες κινούνται πολύ κοντά στην ταχύτητα του φωτός.}, το οποίο αποτελεί επιπλέον πρόκληση για τη χρήση του \en{Electron Beam Scanner}.

Για τη διερεύνηση της επίδρασης των μικρού μήκους δεσμών και για την βελτιστοποίηση του σχεδιασμού του \en{Electron Beam Scanner}, στην παρούσα διπλωματική εργασία γίνεται μια πλήρης ανάλυση,  ανιχνεύεται η κίνηση δεσμών ανίχνευσης μέσα από το ηλεκτρομαγνητικό πεδίο της Δέσμης Οδηγού και των μερών που την περιβάλλουν. 
Στα πλαίσια αυτής της διερεύνησης χρησιμοποιήθηκε η σουίτα προσομοιώσεων \en{CST}, λόγω των δυνατοτήτων της σχετικά με τον συνδυασμό ηλεκτρομαγνητικών υπολογισμών και ανίχνευσης σωματιδίων στις προσομοιώσεις.

Επιπλέον, αναζητήθηκαν ποικίλοι τρόποι για την ανίχνευση και τον προσδιορισμό της δέσμης ανίχνευσης και δημιουργήθηκε ο αντίστροφος αλγόριθμος ανακατασκευής του προφίλ της Δέσμης Οδηγού, σε μια προσπάθεια συνδρομής στο σχεδιασμό ενός επιταχυντή επόμενης γενιάς.

\section{Διάρθρωση}
Η παρούσα διπλωματική εργασία είναι οργανωμένη σε πέντε κεφάλαια:

Στο Κεφάλαιο 2 δίνεται το θεωρητικό υπόβαθρο της εργασίας. 
Αυτό περιλαμβάνει μια σύντομη περιγραφή του \en{CERN}, στο οποίο έγινε το μεγαλύτερο κομμάτι της εργασίας, του \en{CLIC}, που αποτελεί τον γραμμικό επιταχυντή που σχεδιάζεται να δημιουργηθεί σε χρονικό ορίζοντα 15 ετών, και παρουσιάζεται και επεξηγείται η λειτουργία του \en{Electron Beam Scanner}, καθώς επίσης και η μελέτη που έχει γίνει για αυτόν από την επιστημονική κοινότητα ως τώρα.

Στο Κεφάλαιο 3 παρουσιάζονται οι μέθοδοι προσομοίωσης που χρησιμοποιήθηκαν.
Αρχικά, έχουμε μια σύντομη περιγραφή των βασικών εργαλείων που χρησιμοποιήθηκαν, του \en{CST} και του \en{MATLAB}, και στη συνέχεια επεξηγούνται τα στάδια της ανάλυσης που έγινε: ο τρόπος διερεύνησης της επιρροής παραμέτρων του επιταχυντή σε έναν \en{Electron Beam Scanner}, η προσομοίωση της διάταξης εξολοκλήρου στο προσομοιωτικό περιβάλλον του \en{CST} και, τέλος, ο συνδυασμός των δύο εργαλείων, του \en{CST} και του \en{MATLAB}, με σκοπό τη βελτίωση της απόδοσης της προσομοίωσης και της δημιουργίας νέων δυνατοτήτων για προσομοίωση.

Στο Κεφάλαιο 4 παρουσιάζονται και σχολιάζονται τα αποτελέσματα της μελέτης του προηγούμενου κεφαλαίου.

Τέλος, στο Κεφάλαιο 5 δίνονται τα συμπεράσματα, η συνεισφορά αυτής της
διπλωματικής εργασίας, καθώς και προτάσεις για μελλοντική επέκτασή της.

Στο τέλος της διπλωματικής εργασίας βρίσκονται και 2 παραρτήματα.
Το Παράρτημα Α΄ περιέχει τις μεταφράσεις ξένων όρων που χρησιμοποιήθηκαν, ενώ στο Παράρτημα Β΄ περιγράφεται αναλυτικά το μοντέλο που στήθηκε στο \en{CST}, σε περίπτωση που ο αναγνώστης θελήσει να το χρησιμοποιήσει.
Μετά τα 2 παραρτήματα ο αναγνώστης μπορεί να βρει τις λίστες Σχημάτων και Πινάκων και την Περίληψη (\en{Abstract}) της διπλωματικής στην Ελληνική και Αγγλική γλώσσα.