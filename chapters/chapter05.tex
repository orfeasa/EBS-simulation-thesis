\chapter{Επίλογος}
Σε αυτό το κεφάλαιο παρουσιάζονται κάποια συμπεράσματα που μπορούν να εξαχθούν από τη διπλωματική εργασία, καθώς και ενδεχόμενες μελλοντικές επεκτάσεις όσων παρουσιάστηκαν.

\section{Συμπεράσματα}
Τα συμπεράσματα που προέκυψαν από τη διπλωματική είναι τα παρακάτω:
\begin{itemize}
\item Το \en{CST} είναι ένα εργαλείο που μπορεί να αντεπεξέλθει αποτελεσματικά στις ανάγκες της έρευνας στην κατεύθυνση εναλλακτικών τρόπων σάρωσης του προφίλ δεσμών ηλεκτρονίων, με χρήση του εξαιρετικά ισχυρού \en{CST Particle Studio} και του \en{particle-in-cell (PIC) solver}.
\item Για πολλαπλές επαναλαμβανόμενες προσομοιώσεις όπου το \en{CST} θα πρέπει να υπολογίζει ίδιες τροχιές, καθίσταται μη αποδοτικό λόγω περιορισμών του προγράμματος στο να χαρακτηρίζονται τμήματα της προσομοίωσης ως $``$όχι απαραίτητα για υπολογισμό εκ νέου σε κάθε προσομοίωση$"$. 
Για τέτοιες περιπτώσεις η επιστράτευση εργαλείων όπως το \en{MATLAB} διευκολύνει την απόδοση και αυξάνει την ελευθερία διαχείρισης των αποτελεσμάτων στον χρήστη τους.
\item Ο \en{Electron Beam Scanner} αποτελεί έναν τρόπο ανίχνευσης του εγκάρσιου προφίλ δέσμης σωματιδίων ο οποίος, μετά από μια αρχική ανάλυση, φαίνεται να είναι ένας από τους πολλά υποσχόμενους τρόπους μη επεμβατικής ανίχνευσης.
\end{itemize}

\section{Μελλοντικές Επεκτάσεις}
Το σύστημα που αναπτύχθηκε στα πλαίσια αυτής της διπλωματικής εργασίας θα μπορούσε να βελτιωθεί και να επεκταθεί περαιτέρω,
τουλάχιστον ως προς τρεις κατευθύνσεις. 
Συγκεκριμένα:

\begin{enumerate}
\item Βελτιστοποίηση της απόδοσης του μοντέλου στο \en{CST}.

Παρά το γεγονός ότι η διαμόρφωση του τελικού μοντέλου στο \en{CST} είναι αποτέλεσμα πολύμηνης ασχολίας και προσπάθειας συνεχούς βελτιστοποίησης, σε συνεργασία και με την ίδια την ομάδα υποστήριξης του \en{CST}, πάντα υπάρχουν περιθώρια βελτίωσης. 
Συγκεκριμένα, σαν επόμενο βήμα θα βλέπαμε τον εντοπισμό ακριβώς όσων δεδομένων μας είναι χρήσιμα για την εξαγωγή του προφίλ, κατά τη διάρκεια της εκτέλεσης της προσομοίωσης, και την προσαρμογή του μοντέλου έτσι, ώστε τα δεδομένα που δεν μας είναι χρήσιμα να μην υπολογίζονται και να μην αποθηκεύονται.
Αυτό μπορεί να πραγματοποιηθεί με διάφορους τρόπους, όπως τη δημιουργία πυκνότερου και αραιότερου πλέγματος σε άλλα σημεία της προσομοίωσης και με τη εξέταση δημιουργίας επιπλέον \en{macros} σε \en{Visual Basic} στο \en{CST}.

Επιπλέον, το μοντέλο μπορεί να χωριστεί σε δύο ξεχωριστά \en{CST projects}, όπου στο ένα θα προσομοιώνεται η λειτουργία μόνο της κύριας δέσμης, και στο δεύτερο θα εισάγεται αυτό που προσομοιώθηκε στο πρώτο και θα προσομοιώνεται εκεί η λειτουργία της δευτερεύουσας δέσμης.
Κατά το χρόνο συγγραφής της παρούσας διπλωματικής εργασίας η συγκεκριμένη λειτουργία δεν υποστηριζόταν από τον \en{particle-in-cell (PIC) solver}, αλλά υποστηρίζεται από άλλους.
Μετά από επικοινωνία με την ομάδα υποστήριξης του \en{CST}, ενημερωθήκαμε ότι αυτό αποτελεί \en{feature} που έχει προγραμματιστεί να προστεθεί σε επόμενες εκδόσεις του προγράμματος.
\item Βελτιστοποίηση της ταχύτητας εκτέλεσης της προσομοίωσης στο \en{MATLAB}.

Όπως κάθε τύπου προσομοίωσης ή προγράμματος, έτσι και το πρόγραμμα εισαγωγής του ηλεκτρικού πεδίου της κύριας δέσμης από το \en{CST} στο \en{MATLAB} και η προσομοίωσης της δέσμης ανίχνευσης επιδέχεται βελτιώσεων.
Δεδομένου ότι εν τέλει αυτό που μας ενδιαφέρει είναι η μέγιστη απόκλιση της δέσμης ανίχνευσης κατά τον άξονα $Y$, για τις διάφορες αρχικές θέσεις ριπής, μια αναλυτική μελέτη του εισαγόμενου ηλεκτρικού πεδίου μπορεί να μας δώσει πληροφορίες που μπορούμε να χρησιμοποιήσουμε για την προ-επεξεργασία του ηλεκτρικού πεδίου, καθώς και την μείωση των τροχιών που υπολογίζονται μέχρι τέλους, αν αυτές δεν θα αποτελούν $``$υποψήφια τροχιά που θα δώσει μέγιστο $\theta_y$. 

\item Επαλήθευση των αποτελεσμάτων που λαμβάνουμε προσομοιωτικά με πειραματικά αποτελέσματα.

Η μέθοδος της προσομοίωσης είναι εξαιρετικά βοηθητική για την εξαγωγή συμπερασμάτων για το κατά πόσο η μέθοδός που εξετάσαμε είναι αποτελεσματική και υλοποιήσιμη.
Παρόλα αυτά, η δημιουργία πειραματικών διατάξεων θα είναι το επόμενο βήμα για την αξιολόγηση της ακρίβειας της μεθόδου, και της σχέσης ακρίβειας και τιμής, ώστε εν τέλει να παρθεί η απόφαση αν έχει νόημα η επιπλέον έρευνα για τη χρήση του \en{Electron Beam Scanner} σε γραμμικούς επιταχυντές με υψηλές ενέργειες όπως τον \en{CLIC}.
\end{enumerate}