\selectlanguage{greek}

\chapter{Το μοντέλο στο \en{CST Particle Studio}}\label{ch:CSTmodel}
\section{Λίστα παραμέτρων}\label{sec:CSTparameterlist}
Στον παρακάτω πίνακα παρουσιάζονται όλες οι παράμετροι που έχει το τελικό \en{CST project} που χρησιμοποιήθηκε για την πλήρη προσομοίωση του \en{Electron Beam Scanner}.

\begin{longtabu} to \textwidth {
>{\ttfamily}X[3,l]
>{\ttfamily}X[3,c]
X[4,l] }
%\label{table:CST-parameters}
\toprule
Όνομα παραμέτρου	&	Τιμή	&	Περιγραφή  \\ 
\midrule
\endfirsthead

%\label{table:CST-parameters}
\toprule
Όνομα παραμέτρου	&	Τιμή	&	Περιγραφή  \\ 
\midrule
\endhead

\midrule
\multicolumn{3}{r}{Συνεχίζεται στην επόμενη σελίδα} \\
\caption{Λίστα παραμέτρων του περιβάλλοντος προσομοίωσης στο \en{CST}.}\\
\endfoot

\bottomrule
\caption[]{(συνέχεια) Λίστα παραμέτρων του περιβάλλοντος προσομοίωσης στο \en{CST}.}\\
\endlastfoot

\en{simulation\_time}				&	\en{2e-8}						&	\en{Simulation Time} \\
\en{scan\_pipe\_length}				&	\en{0.25}						&	\en{Scan pipe length} \\
\en{scan\_pipe\_diameter}			&	\en{main\_pipe\_diameter}		&	\en{Scan pipe diameter} \\
\en{scan\_monitor\_step}			&	\en{1e-10}						&	\en{Scan beam monitor step width} \\
\en{scan\_monitor\_start\_\+time}	&	\en{0}							&	\en{Scan beam monitor start time} \\
\en{scan\_beam\_vertical\_\+offset}	&	\en{0.01}						&	\en{Scan Beam vertical offset} \\
\en{scan\_beam\_rise\_time}			&	\en{1e-9}						&	\en{Scan beam rise time} \\
\en{scan\_beam\_pulse\_charge}		&	\en{1}							&	\en{Scan beam charge per pulse} \\
\en{scan\_beam\_offset}				&	\en{0}							&	\en{Scan beam offset} \\
\en{scan\_beam\_length}				&	\en{4e-3}						&	\en{Scan beam length (sigma)} \\
\en{scan\_beam\_energy}				&	\en{2e4}						&	\en{Scan beam energy} \\
\en{scan\_beam\_emission\_\+lines}	&	\en{5}							&	\en{Scan beam emission lines (density)} \\
\en{scan\_beam\_diameter}			&	\en{1e-4}						&	\en{Scan beam diameter} \\
\en{scan\_beam\_cutoff}				&	\en{1e-3}						&	\en{Scan beam cutoff length} \\
\en{scan\_beam\_current}			&	\en{1e-6}						&	\en{Scan beam current} \\
\en{scan\_beam\_bunches}			&	\en{1}							&	\en{Scan beam number of bunches} \\
\en{scan\_beam\_bunch\_\+distances}	&	\en{1e-3}						&	\en{Scan beam distance between bunches} \\
\en{pic\_monitor\_xcut}				&	\en{3 / 4 * \+scan\_pipe\_length\+ / 2}	&	\en{X coordinate of PIC 2D monitor} \\
\en{monitor\_step\_width}			&	\en{5e-10}						&	\en{PIC position monitor step width} \\
\en{main\_pipe\_length}				&	\en{5}							&	\en{Main pipe length} \\
\en{main\_pipe\_diameter}			&	\en{0.1}						&	\en{Main pipe diameter} \\
\en{main\_beam\_rise\_time}			&	\en{1e-9}						&	\en{Main beam rise time} \\
\en{main\_beam\_offset}				&	\en{main\_beam\_length\+ * 2.01}	&	\en{Main beam offset} \\
\en{main\_beam\_number\_\+of\_bunches}&	\en{10}							&	\en{Main beam number of bunches} \\
\en{main\_beam\_lines}				&	\en{7}							&	\en{Main beam emission lines (density)} \\
\en{main\_beam\_length}				&	\en{0.15}						&	\en{Main beam bunch length (sigma)} \\
\en{main\_beam\_energy}				&	\en{1e8}						&	\en{Main beam energy} \\
\en{main\_beam\_diameter}			&	\en{1e-2}						&	\en{Main Beam diameter} \\
\en{main\_beam\_cutoff}				&	\en{main\_beam\_length\+ * 2.01}	&	\en{Main beam cutoff length} \\
\en{main\_beam\_current}			&	\en{4.2}						&	\en{Main beam current} \\
\en{main\_beam\_charge\_per\_\+bunch}	&	\en{main\_beam\_charge\+ / 70128}	&	\en{Main beam charge per bunch} \\
\en{main\_beam\_charge}				&	\en{590e-6}						&	\en{Main beam charge per pulse} \\
\en{main\_beam\_bunch\_\+distances}	&	\en{main\_beam\_length\+ * 10}	&	\en{Main beam distance between bunches} \\
\end{longtabu}
\begin{landscape}
\section{\en{Template Based Post Processing}}\label{sec:CSTpostProcessing}

\begin{longtabu} to \linewidth {c>{\ttfamily}XcX[l]X}
%\label{table:CST-postprocessing}
\toprule
 $\#$ & \en{Result name} & \en{Type} & \en{Expression} & \en{Template name} \\ 
\midrule
\endfirsthead

%\label{table:CST-postprocessing}
\toprule
 $\#$ & \en{Result name} & \en{Type} & \en{Expression} & \en{Template name} \\ 
\midrule
\endhead

\midrule
\multicolumn{5}{r}{Συνεχίζεται στην επόμενη σελίδα} \\
\caption{Οι μεταβλητές και ο τρόπος υπολογισμού τους στο \en{Template Based Post Processing} του \en{CST}.} \\
\endfoot

\bottomrule
\caption[]{ (συνέχεια) Οι μεταβλητές και ο τρόπος υπολογισμού τους στο \en{Template Based Post Processing} του \en{CST}.} \\
\endlastfoot


1	& \en{x-average Position}													& \en{1D}	& \en{x-average position }															& \en{Evaluate PIC 2D monitor with average} \\
2	& \en{y-average Position}													& \en{1D}	& \en{y-average position }															& \en{Evaluate PIC 2D monitor with average} \\
3	& \en{z-average Position}													& \en{1D}	& \en{z-average position }															& \en{Evaluate PIC 2D monitor with average} \\
4	& \en{theta\_y}																& \en{1D}	& \en{\src{Atn( (y-average Position - scan\_beam\_vertical\_offset) / pic\_monitor\_xcut) }}	& \en{Mix template results} \\
5	& \en{theta\_z}																& \en{1D}	& \en{\src{Atn( (z-average position - 0) / pic\_monitor\_xcut) }}						& \en{Mix template results} \\
6	& \en{theta\_y\_1D\_xSub}														& \en{1D}	& \en{Extract data in subrange, \src{theta\_y} for  $x \in [\num{5e-9}, 1]$}						& \en{0D or 1D Result from 1D Result} \\
7	& \en{1st bunch theta\_y\_1D\_xSub}											& \en{1D}	& \en{Extract data in subrange, \src{theta\_y} for  $x \in [\num{0.8e-8}, \num{1.3e-8}]$}				& \en{0D or 1D Result from 1D Result} \\
8	& \en{10th bunch theta\_y\_1D\_xSub}											& \en{1D}	& \en{Extract data in subrange, \src{theta\_y} for  $x \in [\num{5.3e-8}, \num{5.8e-8}]$}				& \en{0D or 1D Result from 1D Result} \\
9	& \en{theta\_z\_1D\_xSub}														& \en{1D}	& \en{Extract data in subrange, \src{theta\_z} for  $x \in [\num{5e-9}, 1]$}						& \en{0D or 1D Result from 1D Result} \\
10	& \en{theta\_y\_0D\_GlobalyMax}												& \en{0D}	& \en{Global y-Maximum, \src{theta\_y} }													& \en{0D or 1D Result from 1D Result} \\
11	& \en{1st bunch theta\_y\_1D\_xSub\_0D\_GlobalyMax}								& \en{0D}	& \en{Global y-Maximum, 1st bunch \src{theta\_y\_1D\_xSub}}								& \en{0D or 1D Result from 1D Result} \\
12	& \en{10th bunch theta\_y\_1D\_xSub\_0D\_GlobalyMax}								& \en{0D}	& \en{Global y-Maximum, 10th bunch \src{theta\_y\_1D\_xSub}}									& \en{0D or 1D Result from 1D Result} \\
13	& \en{theta\_y\_1D\_xSub\_0D\_GlobalyMin}										& \en{0D}	& \en{Global y-Minimum, 1st bunch \src{theta\_y\_1D\_xSub}}									& \en{0D or 1D Result from 1D Result} \\
14	& \en{theta\_z\_0D\_GlobalyMax}												& \en{0D}	& \en{Global y-Maximum, \src{theta\_z} }													& \en{0D or 1D Result from 1D Result} \\
15	& \en{theta\_z\_0D\_GlobalyMin}												& \en{0D}	& \en{Global y-Minimum, \src{theta\_z} }													& \en{0D or 1D Result from 1D Result} \\
16	& \en{Ellipse height}														& \en{0D}	& \en{\src{theta\_y\_0D\_GlobalyMax	- 0}}													& \en{Mix template results} \\
17	& \en{Ellipse width}														& \en{0D}	& \en{\src{theta\_z\_0D\_GlobalyMax	- theta\_z\_0D\_GlobalyMin}}								& \en{Mix template results} \\
18	& \en{Ellipse ratio}														& \en{0D}	& \en{\src{Ellipse height / Ellipse width}}												& \en{Mix template results} \\
19	& \en{Ellipse}																& \en{1DC}	& \en{Parametric X-Y plot, X: \src{theta\_z}, Y: \src{theta\_y} }									& \en{0D or 1D Result from 1D Result} \\
20	& \en{Ellipse\_1}															& \en{1DC}	& \en{Parametric X-Y plot, X: \src{theta\_z\_1D\_xSub}, Y: \src{theta\_y\_1D\_xSub} }					& \en{0D or 1D Result from 1D Result} \\
21	& \en{Convert theta\_y\_0D\_GlobalyMax To 1D}									& \en{1D}	& \en{Table Values in Dependence on Parameter \src{scan\_beam\_vertical\_offset} }			& \en{Convert Template Type} \\
22	& \en{Convert 1st bunch theta\_y\_1D\_xSub\_0D\_GlobalyMax To 1D}				& \en{1D}	& \en{Table Values in Dependence on Parameter \src{scan\_beam\_vertical\_offset }}			& \en{Convert Template Type} \\
23	& \en{Convert 10th bunch theta\_y\_1D\_xSub\_0D\_GlobalyMax To 1D}				& \en{1D}	& \en{Table Values in Dependence on Parameter \src{scan\_beam\_vertical\_offset} }			& \en{Convert Template Type} \\
24	& \en{Convert theta\_y\_0D\_GlobalyMax To 1D\_1D\_Deriv}							& \en{1D}	& \en{Derivative, \src{theta\_y\_0D\_GlobalyMax}}											& \en{0D or 1D Result from 1D Result} \\
25	& \en{Convert 1st bunch theta\_y\_1D\_xSub\_0D\_GlobalyMax To 1D\_1D\_Deriv}		& \en{1D}	& \en{Derivative, 1st bunch \src{theta\_y\_1D\_xSub\_0D\_GlobalyMax}}							& \en{0D or 1D Result from 1D Result} \\
26	& \en{Convert 10th bunch theta\_y\_1D\_xSub\_0D\_GlobalyMax To 1D\_1D\_Deriv}		& \en{1D}	& \en{Derivative, 10th bunch \src{theta\_y\_1D\_xSub\_0D\_GlobalyMax}}							& \en{0D or 1D Result from 1D Result} \\
\end{longtabu}
\end{landscape}

\section{Μονάδες Μέτρησης \en{(Units)}}
Στον Πίνακα \ref{table:CST-units} φαίνονται οι μονάδες μέτρησης του περιβάλλοντος στο \en{CST}, όπως αυτές ζητείται να οριστούν.
Οι μονάδες είναι όλες μονάδες του \en{SI}, χωρίς τη χρήση πολλαπλασιαστών (προθεμάτων), εκτός από τη χωρητικότητα και την επαγωγή, τα οποία δε χρησιμοποιήθηκαν στο \en{project} μας.

\begin{table}
\centering
\begin{tabular}{l c}
\toprule
Μέγεθος				& Μονάδα μέτρησης \\
\midrule
\en{Dimentions}		& \en{m} \\
\en{Temperature}	& \en{Kelvin} \\
\en{Frequency}		& \en{Hz} \\
\en{Time} 	 		& \en{s} \\
\en{Voltage}	 	& \en{V} \\
\en{Current}	 	& \en{A} \\
\en{Resistance} 	& \en{Ohm} \\
\en{Conductance}	& \en{S} \\
\en{Inductance} 	& \en{nH} \\
\en{Capacitance}	& \en{pF} \\
\bottomrule
\end{tabular}
\caption{Οι μονάδες μέτρησης του περιβάλλοντος του \en{CST}.}
\label{table:CST-units}
\end{table}

