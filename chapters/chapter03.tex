\chapter{\selectlanguage{greek}Μέθοδοι προσομοίωσης}
Κεφάλαια 3 και 4
\begin{itemize}
\item Αποτέλεσμα και σχόλια
\item Περιγραφή του \en{CST Particle Studio}
\item \en{Screenshots}
\end{itemize}
Στο κεφάλαιο αυτό περιγράφεται η υλοποίηση του συστήματος, με βάση
τη μελέτη που παρουσιάστηκε στο προηγούμενο κεφάλαιο. Αρχικά
παρουσιάζεται η πλατφόρμα και τα προγραμματιστικά εργαλεία που
χρησιμοποιήθηκαν. Στη συνέχεια δίνονται οι λεπτομέρειες υλοποίησης
για τους βασικούς αλγορίθμους του συστήματος καθώς και η δομή του
κώδικα.

\begin{enumerate}
	\item \en{Directory $``$Static beam$"$}. Από \cite{Locatchov1999} έχουμε τις σχέσεις 
		\begin{equation}
			\theta_y (x) = \frac{2 \rho r_e}{\beta} \int_{-\infty}^{\infty}\frac{n(z) \dd z}{\rho^2 + \left(x+\beta z \right) ^2}
		\end{equation}
		\begin{equation}
			\theta_z(x) = 2 r_e \int_{-\infty}^{\infty}\frac{(x+\beta z)n(z) \dd z}{\rho^2 + \left(x+\beta z \right) ^2}
		\end{equation}
		Αυτά τα κάνουμε \en{plot} και είδαμε πώς επηρεάζονται από 
		\begin{itemize}
			\item \en{bunch intensity}
			\item \en{bunch length}
			\item \en{Y-offset} ($\rho$) 
			\item \en{probe beam voltage} 
		\end{itemize}   
\end{enumerate}

\section{\selectlanguage{greek}Προσομοίωση με το \en{CST Particle Studio}}
Το \en{CST Particle Studio} μπλα μπλα μπλα.

\section{\selectlanguage{greek}Προσομοίωση με το \en{CST Particle Studio} και το \en{MATLAB}}

